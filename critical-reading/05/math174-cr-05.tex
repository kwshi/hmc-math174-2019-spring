\documentclass{math174}

\criticalreading{5}
\date{Monday, February 25}

\sections{1.8}
\problems{(1.13) 3, 5 (b).}

\author{}
\coauthor{}

\begin{document}
\begin{description}
\item[(1.13) 3] Let \(G\) act on \(S\) with corresponding permutation
  representation \(\CC \mathbf S\).  Prove the following.

  \begin{enumerate}
  \item The matrices for the action of \(G\) on the standard basis are
    permutation matrices.

    \begin{solution}

    \end{solution}

  \item If the character of this representation is \(\chi\) and
    \(g \in G\), then
    \[
      \chi(g) = \text{the number of fixed points of \(g\) acting on
        \(S\)}.
    \]

    \begin{solution}

    \end{solution}
  \end{enumerate}

\item[(1.13) 5] If \(X\) is a matrix representation of a group \(G\),
  then its \emph{kernel} is the set \(N = \setst{g \in G}{X(g) = I}\).
  A representation is \emph{faithful} if it is one-to-one.
  \begin{enumerate} \setcounter{enumi}{1}
  \item Suppose that \(X\) has character \(\chi\) and degree \(d\).
    Prove that \(g \in N\) if and only if \(\chi(g) = d\).
    \emph{Hint:} show that \(\chi(g)\) is a sum of roots of unity.

    \begin{solution}

    \end{solution}
  \end{enumerate}
\end{description}
\end{document}
