\documentclass{math174}

\criticalreading{5}
\date{Monday, February 25}

\sections{1.8}

\problems{(1.13) 3, 5 (b).}
\plainliststyles

\author{}
\coauthor{}

\begin{document}

\begin{enumerate}
\item
  \begin{description}
  \item[(1.13) 3] Let \(G\) act on \(S\) with corresponding permutation
    representation \(\CC S\). Prove the following.

    \begin{enumerate}
      \item The matrices for the action of \(G\) on the standard basis are
        permutation matrices.
      \item If the character of this representation is \(\chi\) and \(g \in G\),
        then
        \begin{center}
          \(\chi(g)\) = the number of fixed points of \(g\) acting on \(S\).
        \end{center}
    \end{enumerate}

    \begin{solution}
      \begin{enumerate}
        \item
        \item
      \end{enumerate}
    \end{solution}

  \end{description}
\item
  \begin{description}
  \item[(1.13) 5] If \(X\) is a matrix representation of a group \(G\),
    then its \emph{kernel} is the set \(N = \setst{g \in G}{X(g) = I}\).
    A representation is \emph{faithful} if it is one-to-one.

    \begin{enumerate}
      \item[(b)] Suppose that \(X\) has character \(\chi\) and degree \(d\).
        Prove that \(G \in N\) if an only if \(\chi(g) = d\). \emph{Hint:} show
        that \(\chi(g)\) is a sum of roots of unity.
    \end{enumerate}

    \begin{solution}

    \end{solution}
  \end{description}

\end{enumerate}




\end{document}
