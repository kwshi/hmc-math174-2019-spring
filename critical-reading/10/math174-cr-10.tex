\documentclass{../../math174}

\criticalreading{10}
\date{Monday, April 15}

\sections{2.4}
\problems{(2.12) 5 (a), (d)}

\author{}
\coauthor{}

\begin{document}
\begin{problemlist}
\item[2.12.5] Let the group \(G\) act on the set \(S\).  We say that
  \(G\) is \emph{transitive} if, given any \(s, t \in S\), there is a
  \(g \in G\) with \(gs = t\).  The group is \emph{doubly transitive}
  if, given any \(s, t, u, v \in S\) with \(s \ne u\) and \(t \ne v\),
  there is a \(g \in G\) with \(gs = t\) and \(gu = v\).  Show the
  following.
  \begin{enumerate}
  \item The orbits of \(G\)'s action partition \(S\).

    \begin{solution}
      \begin{proof}

      \end{proof}
    \end{solution}

    \setcounter{enumi}{3}

  \item Use part (c) to conclude that in \(\Sgroup_n\) the function
    \[
      f(\pi) = (\text{number of fixed points of \(\pi\)}) - 1
    \]
    is an irreducible character.

    \begin{book}
      \begin{enumerate}[label=(c)]
      \item If \(G\) is doubly transitive and \(V\) has character
        \(\chi\), then \(\chi-1\) is an irreducible character of
        \(G\).
      \end{enumerate}
    \end{book}

    \begin{solution}
      \begin{proof}

      \end{proof}
    \end{solution}
  \end{enumerate}
\end{problemlist}
\end{document}